\documentclass[]{article}

\begin{document}
\section{Bijlage Executing Blokken}
\label{bijlageblok}
\subsection{Variables en Type}
Deze categorie bevat alle blokken met betrekking tot variabelen en hun types. 
\subsubsection{Type blok}
Een type blok toont een bepaalde type aan voor een variabele of parameter. De ondersteunde types zijn: strings, getallen (floating point getallen) en booleans. Lijsten zouden toegevoegd kunnen worden als extra.
\subsubsection{Variabele blok}
De variabele blok cree\"{e}rt een nieuwe lokale variabele van het gegeven type met een gegeven unieke identifier naam.
\subsubsection{Lock}
Het lock blok bevat een variabele die gelocked moet worden.	
\subsubsection{Unlock}
Het unlock blok bevat een variabele die geunlocked moet worden.
\subsubsection{Value}
Het value blok bevat een string dat de waarde voorstelt die de user ingetyped heeft.
\subsubsection{Set blok}
De set blok is een assignment blok. Deze assigned een waarde of de waarde van een variabele aan een andere variabele.
\subsubsection{Var blok}
De var blok heeft een naam waarmee een variable mee kan worden aangesproken of gemanipuleerd.

\subsection{Motion}
Deze categorie bevat alle blokken die een Instantie kunnen laten bewegen in het visuele canvas. 
\subsubsection{Move blok}
De move blok stelt een translatie voor van de instance. De blok laat de instance op de x-as bewegen met [x] stappen en op de y-as bewegen met [y] stappen. Beide waardes zijn floating point getallen.

\subsection{Visual}
Deze categorie bevat alle blokken waarmee een Instantie visueel kan veranderen op het canvas.
\subsubsection{Show}
Het show blok maakt een instance zichtbaar of doet niets indien de instance al zichtbaar was.
\subsubsection{Hide}
Het hide blok maakt een instance onzichtbaar of doet niets indien de instance al onzichtbaar was.
\subsubsection{Change appearance}
Het changeAppearance blok maakt het mogelijk voor een instance om zijn uiterlijk te veranderen.

\subsection{String operators}
Deze categorie bevat alle blokken die te maken hebben met string manipulatie.
\subsubsection{Concat}
Het concat blok laat toe om 2 strings samen te voegen.
\subsubsection{Length}
Het strlen blok laat toe om te lengte van een string op te vragen.
\subsubsection{CharAt}
Het charAt blok laat toe om character op een bepaalde index op te vragen.

\subsection{Operator}
Het operator blok stelt een operator voor zoals +,-,/,*,$<$,$>$,... . 

\subsection{Logic operators}
Deze categorie bevat alle blokken die te maken hebben met logische operaties.
\subsubsection{Unaire logic operator}
Het unLogicOpp" blok stelt een logische operatie voor die unair is. 
\subsubsection{Binaire logic operator}
Het binLogicOpp blok stelt een logische operatie voor die binair is. 

\subsection{Arithmetic blokken}
Deze categorie bevat alle blokken die te maken hebben met arithmische operaties.
\subsubsection{Random}
Het random blok stelt een functie voor die een random gekozen getal teruggeeft tussen de meegegeven bounds. 
\subsubsection{Arith blok}
Een rekenkundige expressie blok berekent een expressie en geeft een number value terug.

\subsection{Functions en Handlers}
Deze categorie bevat alle blokken die gebruikt worden wanneer een gebruiker een functie of handler wilt maken of aanroepen.
\subsubsection{Handler}
Het handler blok stelt een speciale functie voor die een event opvangt. 
\subsubsection{param}
Het param blok stelt een parameter voor van een functie, deze heeft een type en een identifier.
\subsubsection{Function}
Het function blok stelt een functie voor die opgeroepen kan worden in een ander stuk code van deze class. 
\subsubsection{Return blok}
Een blok die gebruikt wordt om \'{e}\'{e}n of meerdere variabelen te returnen van een functie.
\subsubsection{FunctionCall blok}
Een FunctionCall blok stelt een functie oproep voor en bevat variabelen voor de oproep.
De laaste variable is de return waarde als de functie iets returned.

\subsection{Class}
Deze sectie bevat alle blokken die te maken heeft met een Klasse.
\subsubsection{Class}
Het Class blok stelt een volledige class voor. Deze list al zijn input events, alle verschillende emits en alle member variabelen.

\subsection{Control Blokken}
Deze categorie bevat alle blokken die te maken hebben met controle blokken.
\subsubsection{Forever blok}
De forever blok herhaald een stuk code vanaf oproep van de blok tot einde van programma.
\subsubsection{If blok}
De if blok bevat een conditie en een code blok dat wordt uitgevoerd als de conditie waar is.
\subsubsection{If-else blok}
De if-else blok bevat een conditie en twee code blokken. Bij het waar zijn van de conditie wordt de eerste blok code uitgevoerd anders de tweede blok.
\subsubsection{While blok}
De while blok bevat een conditie. Zolang die conditie naar true wordt gevalueerd wordt de code herhaald.

\subsection{Events en Emits}
Deze categorie bevat blokken waarmee de gebruiker events kan gebruiken. 
\subsubsection{Emit blok}
De emit blok verstuurt een event van een bepaald type dat er wordt ingevuld. 
\subsubsection{Event blok}
Een event blok voor het tonen en cree\"{e}ren van een event. Deze bevat een uniek type en members van een specifiek type met een unieke naam in het event.
\subsubsection{Member blok}
De member blok kan een een event zitten. Dit is een variabele dat een type heeft en een naam.
\subsubsection{Access blok}
De access blok bevat een event en de naam van de member die men wilt aanspreken.
Deze geeft deze variable zijn value terug.

\subsection{Instances en Wires}
Deze categorie bevat alle blokken die nodig zijn om een programma flow te maken.
\subsubsection{Instance }
Dit stelt de XML voor een instance op te slaan voor. Een instance heeft een sprite waarbij het behoort een positie en een unieke naam.
\subsubsection{Wire}
Een wire heeft twee instances en het event dat er tussen verstuurd wordt.
\subsubsection{WireFrame}
Een wireFrame bevat instances en de wires tussen die instances.


\section{Bijlage XML Blokken}
\label{bijlagexml}
\subsection{Variables en Type}
\subsubsection{Type blok}
De \texttt{type} blok specifieert een bepaald type bv. een string of getal.
\lstset{language=XML}
\begin{lstlisting}
<type name="name of type" />
\end{lstlisting}
De DOCTYPE declaration: 
\lstset{language=XML}
\begin{lstlisting}
<!ELEMENT type EMPTY>
<!ATTLIST type name CDATA #REQUIRED>
\end{lstlisting}

\subsubsection{Variabele blok}
De \texttt{makeVar} blok heeft een bepaalde unieke naam.	
Dit element bevat een type element.
\lstset{language=XML}
\begin{lstlisting}
<makeVar name="name of variable">
	<type name="number" />
</makeVar>
\end{lstlisting}
De DOCTYPE declaration: 
\lstset{language=XML}
\begin{lstlisting}
<!ELEMENT makeVar (type)>
<!ATTLIST makeVar name CDATA #REQUIRED>
\end{lstlisting}

\subsubsection{Lock}
De \texttt{lock} blok bevat een variabele die gelocked moet worden.	
\lstset{language=XML}
\begin{lstlisting}
<lock>
	<var name="name" />
</lock>
\end{lstlisting}
De DOCTYPE declaration: 
\lstset{language=XML}
\begin{lstlisting}
<!ELEMENT lock (var)>
\end{lstlisting}

\subsubsection{Unlock}
De \texttt{unlock} blok bevat een variabele die geunlocked moet worden.	
\lstset{language=XML}
\begin{lstlisting}
<unlock>
	<var name="name" />
</unlock>
\end{lstlisting}
De DOCTYPE declaration: 
\lstset{language=XML}
\begin{lstlisting}
<!ELEMENT unlock (var)>
\end{lstlisting}

\subsubsection{Value}
De \texttt{value} blok bevat een string die de data voorstelt. 
\lstset{language=XML}
\begin{lstlisting}
<value> value </value>
\end{lstlisting}
De DOCTYPE declaration: 
\lstset{language=XML}
\begin{lstlisting}
<!ELEMENT value (#PCDATA)>
\end{lstlisting}

\subsubsection{Set blok}
De \texttt{setVar} blok bevat een variabele en een value die geassigned zal worden aan deze variabele.
\lstset{language=XML}
\begin{lstlisting}
<setVar>
	<var name="name variable" />
	<value> value </value>
</setVar>
\end{lstlisting}
\lstset{language=XML}
\begin{lstlisting}
<setVar>
	<var name="name variable" />
	<var name="name variable 2" />
</setVar>
\end{lstlisting}
De DOCTYPE declaration: 
\lstset{language=XML}
\begin{lstlisting}
<!ELEMENT setVar (var, value|var|strlen|concat|
			logicOpp|unOpp|binOpp|random|charAt|arith)>
\end{lstlisting}
\subsubsection{var blok}
De \texttt{var} blok heeft een naam waarmee een variable mee kan worden aangesproken of gemanipuleerd.
\lstset{language=XML}
\begin{lstlisting}
<var name="varName"/>

\end{lstlisting}
De DOCTYPE declaration: 
\lstset{language=XML}
\begin{lstlisting}
<!ELEMENT var EMPTY>
<!ATTLIST var name CDATA #REQUIRED>
\end{lstlisting}

\subsection{Motion}
\subsubsection{Move blok}
De \texttt{move} blok stelt een translatie voor.	
\lstset{language=XML}
\begin{lstlisting}
<move xChange="3" yChange="5" />
\end{lstlisting}
De DOCTYPE declaration: 
\lstset{language=XML}
\begin{lstlisting}
<!ELEMENT move EMPTY>
<!ATTLIST move xChange CDATA #IMPLIED yChange CDATA #IMPLIED>
\end{lstlisting}

\subsection{Visual}
\subsubsection{Show}
De \texttt{show} blok maakt een instance zichtbaar of doet niets indien de instance al zichtbaar was.
\lstset{language=XML}
\begin{lstlisting}
<show />
\end{lstlisting}
De DOCTYPE declaration: 
\lstset{language=XML}
\begin{lstlisting}
<!ELEMENT show EMPTY>
\end{lstlisting}

\subsubsection{Hide}
De \texttt{hide} blok maakt een instance onzichtbaar of doet niets indien de instance al onzichtbaar was.
\lstset{language=XML}
\begin{lstlisting}
<hide />
\end{lstlisting}
De DOCTYPE declaration: 
\lstset{language=XML}
\begin{lstlisting}
<!ELEMENT hide EMPTY>
\end{lstlisting}

\subsubsection{Change appearance}
De \texttt{changeAppearance} blok maakt het mogelijk voor een instance om zijn uiterlijk te veranderen. De id is de ID van zijn nieuw uiterlijk.
\lstset{language=XML}
\begin{lstlisting}
<changeAppearance id="0"/>
\end{lstlisting}
De DOCTYPE declaration: 
\lstset{language=XML}
\begin{lstlisting}
<!ELEMENT changeAppearance EMPTY>
<!ATTLIST changeAppearance id CDATA #REQUIRED>
\end{lstlisting}

\subsection{String operators}
\subsubsection{Concat}
De \texttt{concat} blok laat toe om 2 strings samen te voegen.
\lstset{language=XML}
\begin{lstlisting}
<concat>
	<var name="left var to concat">
	<var name="right var to concat">
<concat>
\end{lstlisting}
De DOCTYPE declaration: 
\lstset{language=XML}
\begin{lstlisting}
<!ELEMENT concat (value|var|concat, value|var|concat)>
\end{lstlisting}

\subsubsection{Length}
De \texttt{strlen} blok laat toe om te lengte van een string op te vragen.
\lstset{language=XML}
\begin{lstlisting}
<strlen>
	<var name="string">
<strlen>
\end{lstlisting}
De DOCTYPE declaration: 
\lstset{language=XML}
\begin{lstlisting}
<!ELEMENT strlen (value|var|concat)>
\end{lstlisting}

\subsubsection{CharAt}
De \texttt{charAt} blok laat toe om character op een bepaalde index op te vragen.
\lstset{language=XML}
\begin{lstlisting}
<charAt>
	<value> index </value>
	<var name="string"/>
<strlen>
\end{lstlisting}
De DOCTYPE declaration: 
\lstset{language=XML}
\begin{lstlisting}
<!ELEMENT charAt (var|value, value|var|concat)>
\end{lstlisting}

\subsection{Operator}
De \texttt{operator} blok stelt een operator voor zoals +,-,/,*,$<$,$>$,... . 
\lstset{language=XML}
\begin{lstlisting}
<operator name="+" />
\end{lstlisting}
De DOCTYPE declaration: 
\lstset{language=XML}
\begin{lstlisting}
<!ELEMENT operator EMPTY>
<!ATTLIST operator name CDATA #REQUIRED>
\end{lstlisting}

\subsection{Logic operators}
\subsubsection{Unaire logic operator}
De \texttt{unLogicOpp} blok stelt een logische operatie voor die unair is. 
\lstset{language=XML}
\begin{lstlisting}
<unLogicOpp>
	<var name="varName"/>
	<operator name="not"/>
</unLogicOpp>
\end{lstlisting}
De DOCTYPE declaration: 
\lstset{language=XML}
\begin{lstlisting}
<!ELEMENT unLogicOpp (var|value|unLogicOpp|binLogicOpp|arith, operator)>
\end{lstlisting}

\subsubsection{Binaire logic operator}
De \texttt{binLogicOpp} blok stelt een logische operatie voor die binair is. 
\lstset{language=XML}
\begin{lstlisting}
<binLogicOpp>
	<var name="varName"/>
	<operator name="and"/>
	<var name="varName"/>
</binLogicOpp>
\end{lstlisting}
De DOCTYPE declaration: 
\lstset{language=XML}
\begin{lstlisting}
<!ELEMENT binLogicOpp (var|value|unLogicOpp|binLogicOpp|arith, operator, 
			var|value|unLogicOpp|binLogicOpp|arith)>
\end{lstlisting}

\subsection{Arithmetic blokken}
\subsubsection{Random}
De \texttt{random} blok stelt een functie voor die een random gekozen getal teruggeeft tussen de meegegeven bounds. 
\lstset{language=XML}
\begin{lstlisting}
<random>
	<var name="varName"/>
	<var name="varName"/>
</random>
\end{lstlisting}
De DOCTYPE declaration: 
\lstset{language=XML}
\begin{lstlisting}
<!ELEMENT random (var|value|arith, 
			var|value|arith)>
\end{lstlisting}
\subsubsection{Arith blok}
Een rekenkundige expressie blok berekent een expressie en geeft een number value terug.
\lstset{language=XML}
\begin{lstlisting}
<arith>
  <var name="varName"/>
  <operator name="+" /> 
  <var name="varName"/>
</arith>
\end{lstlisting}
De DOCTYPE declaration: 
\lstset{language=XML}
\begin{lstlisting}
<!ELEMENT arith (var|value|arith,operator,var|value|arith)>
\end{lstlisting}
\subsection{Functions en Handlers}
\subsubsection{Handler}
De \texttt{handler} blok stelt een speciale functie voor die een event opvangt. 
\lstset{language=XML}
\begin{lstlisting}
<handler name="name" event="type of event">
	<block>
		code
	</block>
</handler>
\end{lstlisting}
De DOCTYPE declaration: 
\lstset{language=XML}
\begin{lstlisting}
<!ELEMENT handler (block)>
<!ATTLIST handler name CDATA #required event CDATA #IMPLIED>
\end{lstlisting}

\subsubsection{Param}
De \texttt{param} blok stelt een parameter voor van een functie, deze heeft een type en een identifier 
\lstset{language=XML}
\begin{lstlisting}
<param type="string" name="name1"/>

\end{lstlisting}
De DOCTYPE declaration: 
\lstset{language=XML}
\begin{lstlisting}
<!ELEMENT param EMPTY>
<!ATTLIST param type CDATA #REQUIRED name CDATA #REQUIRED>
\end{lstlisting}

\subsubsection{Function}
De \texttt{function} blok stelt een functie voor die opgeroepen kan worden in een ander stuk code van deze class. 
\lstset{language=XML}
\begin{lstlisting}
<function name="name">
	<param type="string" name="name1"/>
	<param type="string" name="name2"/>
	<block>
		code
	</block>
</function>
\end{lstlisting}
De DOCTYPE declaration: 
\lstset{language=XML}
\begin{lstlisting}
<!ELEMENT function (param*, block)>
<!ATTLIST function name CDATA #REQUIRED>
\end{lstlisting}
\subsubsection{Return Blok}
Een \texttt{return} blok bevat variabelen die hij returned. Volgorde is hier van belang.
\begin{lstlisting}
\lstset{language=XML}
<return>
	<var name="varName" />
</return>	
\end{lstlisting}
De DOCTYPE declaration: 
\lstset{language=XML}
\begin{lstlisting}
<!ELEMENT return (var)*>
\end{lstlisting}
\subsubsection{FunctionCall blok}
Een \texttt{FunctionCall} blok stelt een functie oproep voor en bevat variabelen voor de oproep.
De laaste variable is de return waarde als de functie iets returned.
\lstset{language=XML}
\begin{lstlisting}
<functionCall name="functioName">
  <var name="varName1"/>
  <var name="varName2"/>
</functionCall>
\end{lstlisting}
De DOCTYPE declaration: 
\lstset{language=XML}
\begin{lstlisting}
<!ELEMENT functionCall (var)*>
<!ATTLIST functionCall name CDATA #REQUIRED>
\end{lstlisting}

\subsection{Block}
De \texttt{block} blok stelt een groepering van code voor. 
\lstset{language=XML}
\begin{lstlisting}
<block>
	code
</block>
\end{lstlisting}
De DOCTYPE declaration: 
\lstset{language=XML}
\begin{lstlisting}
<!ELEMENT block (makeVar|setVar|move|show|hide|changeAppearance|if|if-else|wait|
			repeat|forever|emit|while|functionCall)*>
<!ATTLIST block name CDATA #REQUIRED>
\end{lstlisting}

\subsection{Class}
\subsubsection{InputEvent}
De \texttt{inputEvent} blok is een binnenkomende event van een class.
\lstset{language=XML}
\begin{lstlisting}
<inputEvent type="ev1"/>
\end{lstlisting}
De DOCTYPE declaration: 
\lstset{language=XML}
\begin{lstlisting}
<!ELEMENT inputEvent EMPTY>
<!ATTLIST inputEvent type CDATA #REQUIRED>
\end{lstlisting}

\subsubsection{OutputEvent}
De \texttt{outputEvent} blok is een uitgaande event van een class.
\lstset{language=XML}
\begin{lstlisting}
<outputEvent type="ev2"/>
\end{lstlisting}
De DOCTYPE declaration: 
\lstset{language=XML}
\begin{lstlisting}
<!ELEMENT outputEvent EMPTY>
<!ATTLIST outputEvent type CDATA #REQUIRED>
\end{lstlisting}

\subsubsection{Events}
De \texttt{events} blok is een collectie voor alle functions van een class.
\lstset{language=XML}
\begin{lstlisting}
<events>
	<inputEvent type="ev1"/>
</events>
\end{lstlisting}
De DOCTYPE declaration: 
\lstset{language=XML}
\begin{lstlisting}
<!ELEMENT events (inputEvent)*>
\end{lstlisting}

\subsubsection{Emits}
De \texttt{emits} blok is een collectie voor alle emits die een class kan doen.
\lstset{language=XML}
\begin{lstlisting}
<emits>
	<outputEvent type="ev2"/>
</emits>
\end{lstlisting}
De DOCTYPE declaration: 
\lstset{language=XML}
\begin{lstlisting}
<!ELEMENT emits (outputEvent)*>
\end{lstlisting}

\subsubsection{Handlers}
De \texttt{handlers} blok is een collectie voor alle handlers van een class.
\lstset{language=XML}
\begin{lstlisting}
<handlers>
	<handler name="hand" event="ev1">
		code
	<\handler>
</handlers>
\end{lstlisting}
De DOCTYPE declaration: 
\lstset{language=XML}
\begin{lstlisting}
<!ELEMENT handlers (handler)*>
\end{lstlisting}

\subsubsection{Functions}
De \texttt{functions} blok is een collectie voor alle functions van een class.
\lstset{language=XML}
\begin{lstlisting}
<functions>
	<function name="func">
		code
	<\function>
</functions>
\end{lstlisting}
De DOCTYPE declaration: 
\lstset{language=XML}
\begin{lstlisting}
<!ELEMENT functions (function)*>
\end{lstlisting}

\subsubsection{MemberVariables}
De \texttt{memberVariables} blok is een collectie voor alle member variables van een class.
\lstset{language=XML}
\begin{lstlisting}
<memberVariables>
	<member type="number" name="var1" />
</memberVariables>
\end{lstlisting}
De DOCTYPE declaration: 
\lstset{language=XML}
\begin{lstlisting}
<!ELEMENT memberVariables (member)*>
\end{lstlisting}

\subsubsection{Class}
De \texttt{class} blok stelt volledige class voor. Deze list al zijn input events, alle verschillende emits, alle member variabelen, alle handler functions en alle gewone functions.
\lstset{language=XML}
\begin{lstlisting}
<class name="name">
	<events>
		<inputEvent type="ev1/">
	</events>
	<emits>
		<outputEvent type="ev2"/>
	</emits>
	<handlers>
		<handler name="hand" event="ev1">
			code
		<\handler>
	</handlers>
	<functions>
		<function name="func">
			code
		<\function>
	</functions>
	<memberVariables>
		<member type="number" name="var1" />
	</memberVariables>
</class>
\end{lstlisting}
De DOCTYPE declaration: 
\lstset{language=XML}
\begin{lstlisting}
<!ELEMENT class (events, emits, handlers, functions)>
<!ATTLIST class name CDATA #REQUIRED>
\end{lstlisting}


\subsection{Control Blokken}
\subsubsection{Forever block}
De \texttt{forever} block bevat een block code dat wordt uitgevoerd.
\lstset{language=XML}
\begin{lstlisting}
<forever>
  <block> code </block>
</forever>
\end{lstlisting}
De DOCTYPE declaration: 
\lstset{language=XML}
\begin{lstlisting}
<!ELEMENT forever (block)>
\end{lstlisting}
\subsubsection{If blok}
De \texttt{if} blok bevat een conditie en een code blok.
\lstset{language=XML}
\begin{lstlisting}
<if>
  <cond> condition </cond>
  <block> code </block>
</if>
\end{lstlisting}
De DOCTYPE declaration: 
\lstset{language=XML}
\begin{lstlisting}
<!ELEMENT if (cond,block)>
\end{lstlisting}
\subsubsection{If-else blok}
De \texttt{if-else} blok bevat een conditie en twee code blokken.
\lstset{language=XML}
\begin{lstlisting}
<if-else>
  <cond> condition </cond>
  <block> if-code </block>
  <block> else-code </block>
</if-else>
\end{lstlisting}
De DOCTYPE declaration: 
\lstset{language=XML}
\begin{lstlisting}
<!ELEMENT if-else (cond,block,block)>
\end{lstlisting}
\subsubsection{Conditie blok}
De \texttt{conditie} bevat een variable of een logische expressie.
\lstset{language=XML}
\begin{lstlisting}
<cond>
  <var name="varName"/>
</cond>
\end{lstlisting}
De DOCTYPE declaration: 
\lstset{language=XML}
\begin{lstlisting}
<!ELEMENT  cond (var|unLogicOpp|binLogicOpp)>
\end{lstlisting}
\subsubsection{While blok}
De \texttt{while} bevat een conditie en een code blok.
\lstset{language=XML}
\begin{lstlisting}
<while>
  <cond> condition </cond>
  <block> code </block>
</while>
\end{lstlisting}
De DOCTYPE declaration: 
\lstset{language=XML}
\begin{lstlisting}
<!ELEMENT while (cond,code)>
\end{lstlisting}

\subsection{Events en Emits}
\subsubsection{Emit blok}
Het \texttt{emit} block bevat de naam event en de members van van de message van dit event.
De members zijn variabelen en de volgorde komt overeen met de volgorde van de members van het event.
\lstset{language=XML}
\begin{lstlisting}
<emit eventName="event">
  <var name="var1">
  <var name"var2">
</emit>
\end{lstlisting}
De DOCTYPE declaration: 
\lstset{language=XML}
\begin{lstlisting}
<!ELEMENT emit (var)*>
<!ATTLIST emit eventName CDATA #REQUIRED>
\end{lstlisting}
\subsubsection{Event blok}
Een \texttt{event} blok voor het tonen en cree\"{e}ren van een event. Deze bevat een uniek type en members van een specifiek type met een unieke naam in het event.
\lstset{language=XML}
\begin{lstlisting}
<event type="eventName">
  <member type="memberType" name"memberName"/>
  <member type="memberType2" name"memberName2"/>
</event>
\end{lstlisting}
De DOCTYPE declaration: 
\lstset{language=XML}
\begin{lstlisting}
<!ELEMENT event (member)*>
<!ATTLIST event type CDATA #REQUIRED>
\end{lstlisting}
\subsubsection{Member blok}
De \texttt{member} blok kan een een event zitten. Dit is een variabele dat een type heeft en een naam.
\lstset{language=XML}
\begin{lstlisting}
<member type="memberType" name="memberName"/>
\end{lstlisting}
De DOCTYPE declaration: 
\lstset{language=XML}
\begin{lstlisting}
<!ELEMENT member EMPTY>
<!ATTLIST member type CDATA #REQUIRED name CDATA #REQUIRED>
\end{lstlisting}
\subsubsection{Access blok}
De \texttt{access} blok bevat een event en de naam van de member die men wilt aanspreken.
Deze geeft deze variable zijn value terug.
\lstset{language=XML}
\begin{lstlisting}
<access event="eventName" name="memberName"/>
\end{lstlisting}
De DOCTYPE declaration: 
\lstset{language=XML}
\begin{lstlisting}
<!ELEMENT access EMPTY>
<!ATTLIST access event CDATA #REQUIRED name CDATA #REQUIRED >
\end{lstlisting}

\subsection{Instances en Wires}
\subsubsection{Instance }
Dit stelt de XML voor een instance op te slaan voor. Een instance heeft een class waarbij het behoort een positie en een unieke naam.
\lstset{language=XML}
\begin{lstlisting}
<instance name="instanceName" class="className" x="X" y="Y" />
\end{lstlisting}
De DOCTYPE declaration: 
\lstset{language=XML}
\begin{lstlisting}
<!ELEMENT instance EMPTY>
<!ATTLIST instance event CDATA #REQUIRED name CDATA #REQUIRED sprite			 CDATA #REQUIRED  x CDATA #REQUIRED y CDATA #REQUIRED>
\end{lstlisting}
\subsubsection{Wire}
Een wire heeft twee instances en het event dat er tussen verstuurd wordt.
\lstset{language=XML}
\begin{lstlisting}
<instance from="instanceName" to="instanceName2" event="eventName" />
\end{lstlisting}
De DOCTYPE declaration: 
\lstset{language=XML}
\begin{lstlisting}
<!ELEMENT wire EMPTY>
<!ATTLIST wire from CDATA #REQUIRED to CDATA #REQUIRED event CDATA #REQUIRED >
\end{lstlisting}
\subsubsection{WireFrame}
Een wireFrame bevat instances en de wires tussen die instances.
\begin{lstlisting}
\lstset{language=XML}
<wireFrame>
	<instance name="instance1" class="className" x="1" y="1"/>
	<instance name="instance2" class="className2" x="1" y="1"/>
	<wire from="instance1" to="intance2" event="eventName"/>
</wireFrame>	
\end{lstlisting}
De DOCTYPE declaration: 
\lstset{language=XML}
\begin{lstlisting}
<!ELEMENT wireFrame (instance|wire)*>
\end{lstlisting}
\end{document}